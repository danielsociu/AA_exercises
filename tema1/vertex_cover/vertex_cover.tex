\documentclass[a4paper]{report} 
\usepackage{listings}
\usepackage{amsmath,amssymb}
\usepackage{tikz}
\renewcommand{\thesection}{\arabic{section}}
\renewcommand{\thesubsection}{\thesection.\alph{subsection}}
\title{Tema 1 Algoritmi Avansati}
\author{Sociu Daniel}
\begin{document}

\chapter*{Vertex Cover}
\addcontentsline{toc}{chapter}{Vertex Cover}
\setcounter{section}{0}

\section{Problema 1}

Deci avem $X=\{x_{1},x_{2}\dots x_{n}\}$ multimea variabilelor de tip bool si $C=\{C_{1},C_{2}\dots C_{m}\}$ cele m clauze si algoritmul:

\begin{tabbing}
\hspace{2em}
\= \kill Greedy-3CNF($C$, $X$)\\
1: $C=\{C_{1}\dots C_{m}\}$ mulțimea de predicate, $X=\{x_{1}\dots x_{n}\}$ - mulțime de variabile\\ 
2: cât timp $C\neq \emptyset$ execută\\ 
\> 3: Alegem aleator $C_{j}\in C$.\\
\> 4: Fie $x_{i}$ una dintre variabilele din $C_{j}$ .\\
\> 5: $x_{i}\leftarrow$ true.\\
\> 6: Eliminăm din $C$ toate predicatele ce îl conțin pe $x_{i}$ .\\
7: return $X$\\
\end{tabbing}


\subsection{}

In cazul algoritmului nostru consideram ca $\exists$ o var $x_{1}$ care se afla in toate $C_{i}, i=\overline{1,m}$ si in rest valori unice $x_{C_{i1}}$ si $x_{C_{i2}}$ pentru fiecare $C_{i}, i=\overline{1,m}$.

In acest caz algoritmul nostru poate sa selecteze toate variabilele ignorand $x_{1}$, deci problema care are $OPT=1$, algoritmul nostru poate fi $ALG=2m$, adica nu are o aproximare fixa, si poate lua valori foarte mari.\\

Deci factorul de aproximare worst case este $ALG=2mOPT$

\subsection{}

Pentru a obtine un algoritm 3-aproximativ trebuie sa modificam linia 4, sa faca toate valorile $x_{i}=1, x_{i} \in C_{j}$

\begin{tabbing}
\hspace{2em}\= \kill 
Greedy-3CNF($C$, $X$)\\
1: $C=\{C_{1}\dots C_{m}\}$ mulțimea de predicate, $X=\{x_{1}\dots x_{n}\}$ - mulțime de variabile\\ 
2: cât timp $C\neq \emptyset$ execută\\ 
\> 3: Alegem aleator $C_{j}\in C$.\\
\> 4: Pentru fiecare $x_{i} \in C_{j}$ .\\
\hspace{2em}\= \hspace{2em}\=)\kill 
\> \> 5: $x_{i}\leftarrow$ true.\\
\> \> 6: Eliminăm din $C$ toate predicatele ce îl conțin pe $x_{i}$ .\\
7: return $X$\\
\end{tabbing}

Acum trebuie sa demonstram ca $ALG\leq 3OPT$

La fiecare pas al algoritmului retinem in $P_{i}$ clauzele scoase din $C$. Sa presupunem ca sunt k pasi.\\

Deci $P_{i}=\{C_{j}\in C \mid C_{j}\text{ a fost scos la pasul i}\}$ 

Notam $P=\{P_{i} \mid i=\overline{1,k}\}$

Este evident ca $P$ este o partitie peste $C$, adica $\nexists$ 2 clauze comune intre oricare $P_{i}, P_{j}$ si $\bigcup\limits_{i=1}^{k} P_{i}=C$.

Deci observam ca intre oricare 2 multimi $P_{i}, P_{j}; \exists C_{a}\in P_{i} \text{ si } C_{b}\in P_{j}$ astfel incat $C_{a} \text{ si } C_{b}$ nu au niciun x in comun.

Asadar, optimul $OPT\geq \vert P\vert$ deoarece in cel mai bun caz $\forall P_{i}\in P$ necesita doar un $x_{i}=1$.

Evdent $ALG=3\cdot \vert P \vert$ (adica 3* numarul de pasi al algoritmului in care facem cate 3 variabile 1).

Asadar

\[ALG=3\cdot \vert P\vert \leq 3OPT \implies ALG\leq 3OPT\]

\subsection{}
Ca sa traducem in programare liniara considerm urmatoarea situatie in programare liniara 0/1, care e chivalenta cu problema 3CNF:

\begin{tabbing}
\hspace{2em}\= \kill 
minimizam $\sum_{i=1}^{n}x_{i}$ \\
a.i. \> $x_{a}+x_{b}+x_{c}\geq 1 \text{ unde } x_{a},x_{b},x_{c}\in C_{i}, i=\overline{1,m}$ \\
\> $x_{i}\in \{0, 1\}$
\end{tabbing}
Notam $OPT_{lin_{0/1}}$ rezultatul acestei probleme de programare liniara 0/1 si fie $OPT$ optimul pentru 3CNF. Observam ca:

\[OPT_{lin_{0/1}}=OPT\]

Dar problema de programare liniara 0/1 prezentata mai sus este NPC. Deci vom face o relaxare a problemei pentru a o duce in programare liniara cu numere reale. 
Notam cu $x'_{i}$ valoarea lui $x_{i}$ dusa in intervalul $[0,1]$, adica $x'_{i}\in [0,1]$

Deci acum avem:

\begin{tabbing}
\hspace{2em}\= \kill 
minimizam $\sum_{i=1}^{n}x'_{i}$ \\
a.i. \> $x'_{a}+x'_{b}+x'_{c}\geq 1 \text{ unde } x_{a},x_{b},x_{c}\in C_{i}, i=\overline{1,m}$ \\
\> $0\leq x'_{i}\leq 1$
\end{tabbing}

Fie $OPT_{lin}$ valoarea sumei $\sum_{i=1}^{n}x'_{i}$ in urma rezolvarii problemei de programare liniara prezentata mai sus.
Observam ca $OPT_{lin}\leq OPT_{lin_{0/1}}$ deoarece noi am facut o relaxare a constrangerilor variantei 0/1, adica este cel putin la fel de buna ca ea.

Adica cum $OPT_{lin_{0/1}}=OPT$, obtinem:

\[OPT_{lin}\leq OPT\]

Acum avem algoritmul pentru problema 3CNF:

\begin{tabbing}
Prog-liniara-3CNF($C$, $X$)\\
1: $X=\{x_{1}\dots x_{n}\}$ - mulțime de variabile, $X'=\{x'_{1}\dots x'_{n}\}$ - variabilele $\in [0,1]$ \\ 
2: Rezolva urmatoarea problema de programare liniara:\\
\hspace{2em}\= \kill 
3: \> minimizam $\sum_{i=1}^{n}x'_{i}$ \\
\hspace{2em}\=\hspace{2em}\= \kill 
\> a.i. \> $x'_{a}+x'_{b}+x'_{c}\geq 1 \text{ unde } x_{a},x_{b},x_{c}\in C_{i}, i=\overline{1,m}$ \\
\> \> $0\leq x'_{i}\leq 1$\\
\hspace{2em}\=\kill 
4: $X=\{x_{i}=1 \mid x'_{i}\geq \frac{1}{3}\}$\\
5: return X
\end{tabbing}

\subsection{}

In primul rand trebuie sa demonstram ca toate clauzele $C_{i}\in C$ sunt satisfacute.

Acest lucru reiese din constrangerea $x'_{a}+x'_{b}+x'_{c}\geq 1$, adica cel putin o variabila este $\geq \frac{1}{3}$
adica ia o valoare finala 1, asta pentru fiecare $C_{i}\in C$. Deci algoritmul rezolva corect problema 3CNF.

Acum trebuie sa demonstram ca $ALG\leq 3OPT$

Este evident ca $ALG=\sum_{i=1}^{n}x_{i}$\\

Deci avem ca \[OPT_{lin}=\sum_{i=1}^{n}x'_{i}\]

Si stim ca $OPT_{lin}\leq OPT$ si $x'_{i}\geq \frac{1}{3}$ pentru orice $x_{i}$ care are valoarea 1 la final. Deci:

\[ALG=\sum_{i=1}^{n}x_{i}\leq 3\cdot \sum_{i=1}^{n}x'_{i}\leq 3\cdot OPT_{lin} \leq 3\cdot OPT\]

Deci avem ca algoritmul nostru este 3-aproximativ:

\[ALG\leq 3OPT\]

\end{document}

